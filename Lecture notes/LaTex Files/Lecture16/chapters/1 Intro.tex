\section{مقدمه}

یک درخت تصمیم‌گیری\LTRfootnote{decision tree} یک ساختار داده‌ی سلسله مراتبی است که از استراتژی تجزیه و غلبه\LTRfootnote{divide-and-conquer strategy} استفاده می‌کند.
درخت‌های تصمیم‌گیری از روش‌های غیرپارامتری\LTRfootnote{nonparametric} به شمار می‌آیند و هم برای مسائل دسته‌بندی\LTRfootnote{classification} که هم برای مسائل رگرسیون\LTRfootnote{regression} قابل استفاده می‌باشند. یکی از ویژگی‌های جالب درخت‌های تصمیم قابلیت تفسیر‌پذیری آن‌هاست. در واقع هر درخت تصمیم را می‌توانیم به سادگی به تعدادی قوانین ساده و قابل فهم تبدیل کنیم. از طرفی درخت‌های تصمیم برای یادگیری زمان زیادی صرف نمی‌کنند و می‌توان از آن‌ها برای مجموعه داده‌های بزرگ استفاده کرد. همچنین از آنجایی که درخت‌های تصمیم قابلیت پشتیبانی از داده‌های عددی\LTRfootnote{numerical data} و دسته‌ای\LTRfootnote{categorical data} را دارند، برای آموزش آن‌ها به پیش‌پردازش پیچیده‌ای نیاز نداریم.

از آنجایی که درختان تصمیم ارتباط نزدیکی با نظریه اطلاعات\LTRfootnote{Information Theory} دارند ابتدا نگاهی به مفاهیم کلیدی در این زمینه می‌پردازیم.